%
% Ashta Padi Demo Paper for ISCLS 2026
%
% Demonstration submission (2 pages max)
%

\documentclass[11pt]{article}
\usepackage{scl}
\usepackage{times}
\usepackage{url}
\usepackage{latexsym}
\usepackage{graphicx}

\title{Ashta Padi: A Profession-Based Interactive Platform for \\Yoga Sutras Learning}

\author{Dr. Bhagyashree Joshi \\
  [Your Institution] \\
  [Your Address] \\
  {\tt bhagyashree.joshi@domain.com}
  \and
  Divyangana Kothari \\
  [Your Institution] \\
  [Your Address] \\
  {\tt divyangana.kothari@domain.com}}


\date{}

\begin{document}
\maketitle

\begin{abstract}
We present \textit{Ashta Padi} (``Eight Steps''), a comprehensive learning framework for Indian scriptures that integrates profession-based personalization with university certification pathways. The system guides learners from profession-specific terminology (e.g., \textit{Artha\'{s}\={a}stra} for economists) through Sanskrit grammar fundamentals to advanced scripture study, with structured progression from certificate to PhD level. Unlike traditional approaches, our model automatically selects five relevant scriptures based on the learner's profession, assigns mentors (university professors or private Acharyas), and integrates with university examination systems to provide formal credentials. The current demonstration implements the foundational steps (profession selection, glossary building, grammar lessons) with five annotated Yoga Sutras, while the complete architecture supports mentor-guided learning, live/recorded classes, and career pathway integration. This work demonstrates how computational tools can create structured, credentialed pathways for Sanskrit scholarship accessible to diverse professional communities.
\end{abstract}

\section{Introduction}

The Yoga Sutras of Patanjali (\textit{Yogas\=utras}) remain one of the most influential Sanskrit texts, yet their accessibility to modern learners is limited by linguistic barriers and the need for extensive grammatical knowledge. While digital Sanskrit resources exist~\cite{huet2003zen,goyal2012sanskrit}, few address the pedagogical challenge of personalized learning pathways for diverse professional audiences.

We introduce \textit{Ashta Padi}, an eight-step pedagogical model that combines profession-based content curation with interactive philological analysis. The name references both the traditional eight-limbed path (\textit{a\d{s}\d{t}\={a}\.{n}ga}) of Patanjali's yoga and our eight-stage learning framework.

\section{The Ashta Padi Model}

Our pedagogical framework provides a structured pathway from beginner to advanced Sanskrit scholarship, integrating with formal university certification. The eight stages (Figure~\ref{fig:model}) are:

\begin{enumerate}
\item \textbf{Profession Selection}: Learners identify their professional domain (economist, yoga practitioner, wellness coach, philosopher, psychologist). This exploits cognitive anchoring---humans retain terminology relevant to their interests more effectively.

\item \textbf{Profession-Specific Glossary \& Testing}: The system presents Sanskrit terms specific to the learner's field (e.g., \textit{Artha\'{s}\={a}stra} vocabulary for economists) with interactive assessments to build foundational domain knowledge.

\item \textbf{General Sanskrit Grammar \& Certification}: Comprehensive grammar instruction covering sandhi, \textit{sam\={a}sa}, declensions, and conjugations. Learners take proficiency tests leading to \textbf{Grammar Certification} (Exit Point 1), enabling those seeking only linguistic competency to complete their studies.

\item \textbf{Scripture Selection \& Mentor Assignment}: Based on profession and test performance, the system automatically selects five relevant scriptures (easy to difficult) and assigns a mentor---either a university professor (with offline examination option for Bachelor's degree credit) or a private Acharya for personalized learning.

\item \textbf{Scripture Structure \& Translation}: Detailed explanation of textual organization (s\={u}tra vs. \'{s}loka formats) with translations in the learner's native language. Corresponds to Master's-level study depth.

\item \textbf{Teacher-Guided Learning Path}: Mentor evaluates test results and recommends learning modality: live classes with listed Acharyas/professors, recorded lecture series, or self-paced study. Learners choose between PhD-track intensive study or continued exploratory learning.

\item \textbf{University-Integrated Certification}: Progressive credentialing tied to scripture completion: one scripture (6 months) yields a \textbf{Certificate}, two scriptures (1 year) yield a \textbf{Diploma}, and all five scriptures with university examinations yield a \textbf{Bachelor's degree}. Integration with university platforms enables formal transcript recording.

\item \textbf{Career \& Research Pathways}: Upon completion, learners access curated options: enroll in advanced courses, apply for employment in relevant fields (teaching, wellness, research), or pursue further academic research (Master's/PhD programs).
\end{enumerate}

\begin{figure}[t]
\centering
\includegraphics[width=0.48\textwidth]{complete_ashta_padi_model.png}
\caption{The complete Ashta Padi eight-step pedagogical model with university-integrated certification pathways, mentor assignment, and career progression.}
\label{fig:model}
\end{figure}

\section{System Architecture \& Current Demo}

The complete Ashta Padi system architecture comprises three layers: (1) \textbf{Content Layer} with expert-annotated scripture databases, (2) \textbf{Pedagogy Layer} with adaptive learning algorithms and mentor matching, and (3) \textbf{Integration Layer} connecting to university examination and credentialing systems.

\textbf{Current Demo Scope}: The working prototype implements an enhanced version of Steps 1--3 as a responsive web application using HTML5, CSS3, and JavaScript with JSON data storage. Key features include:

\textbf{Vocabulary Pre-Learning}: Users first encounter 6 key Sanskrit terms (e.g., \textit{yoga}, \textit{citta}, \textit{vṛtti}) through interactive flashcards before studying sutras. Each term includes root etymology, multiple contextual meanings, and example usage from actual sutras. This builds familiarity and reduces cognitive load during scripture study.

\textbf{Dual-Panel Interface}: The main learning screen features a split layout---sutra content on the left (60\%) and grammar reference on the right (40\%). This eliminates context-switching: learners can reference grammar rules (sandhi, compounds, case endings) while studying sutras without navigating away.

\textbf{Collapsible Grammar Sidebar}: Five foundational grammar lessons (\textit{Vyākaraṇa}, \textit{Varṇa}, \textit{Sandhi}, \textit{Pada}, \textit{Dhātu}) are accessible via an accordion interface. Each lesson includes Sanskrit terminology, definitions, and examples. Users expand only the lessons they need, maintaining focus.

\textbf{Practice Assessment}: A 10-question quiz validates vocabulary retention and grammar comprehension, providing instant feedback with explanations. This demonstrates the complete learning cycle: vocabulary acquisition → contextual study → assessment.

\textbf{Annotation Schema}: Each sutra includes Devanagari text, IAST transliteration (e.g., \textit{atha yog\={a}nu\'{s}\={a}sanam}), translation, and word-level grammatical metadata (root, part of speech, case, number).

\textbf{Modular Design}: Vocabulary and grammar modules are reusable across scriptures, enabling efficient scaling. The architecture supports future integration of mentor assignment systems (Step 4), live/recorded class platforms (Step 6), and university APIs for credential issuance (Step 7).

\textbf{Interactive UI}: Collapsible sections, modal dialogs for grammar/vocabulary deep dives, and progress tracking visualizations.

\section{Demonstration Content}

Our demo includes five sutras from the Samadhi Pada (1.1--1.5), covering foundational concepts:

\begin{itemize}
\item \textit{atha yog\={a}nu\'{s}\={a}sanam} (1.1): ``Now, the teaching of Yoga begins''
\item \textit{yoga\'{s} citta-v\d{r}tti-nirodha\d{h}} (1.2): ``Yoga is the cessation of mental fluctuations''
\item \textit{tad\={a} dra\d{s}\d{t}u\d{h} svar\={u}pe 'vasth\={a}nam} (1.3): ``Then the seer abides in its true nature''
\item \textit{v\d{r}tti-s\={a}r\={u}pyam itaratra} (1.4): ``Otherwise, identification with mental patterns''
\item \textit{v\d{r}ttaya\d{h} pa\~{n}catayyah kli\d{s}\d{t}\={a}kli\d{s}\d{t}\={a}\d{h}} (1.5): ``Mental patterns are fivefold, afflicted or non-afflicted''
\end{itemize}

Each sutra includes 15--20 annotated vocabulary terms and 2--3 grammar lessons. The vocabulary for sutra 1.2, for example, includes:
\begin{itemize}
\item \textit{yoga}: union, discipline, practice (root: \textit{yuj}, ``to join'')
\item \textit{citta}: mind-field, consciousness (root: \textit{cit}, ``to perceive'')
\item \textit{v\d{r}tti}: fluctuation, modification (root: \textit{v\d{r}t}, ``to turn'')
\item \textit{nirodha}: cessation, restraint (\textit{ni} + \textit{rudh}, ``to obstruct'')
\end{itemize}

Grammar lessons demonstrate sandhi rules with interactive quizzes. For instance, the combination \textit{yoga} + \textit{anu\'{s}\={a}sanam} = \textit{yog\={a}nu\'{s}\={a}sanam} illustrates the \textit{a} + \textit{a} $\rightarrow$ \textit{\={a}} vowel sandhi rule.

\section{Implementation Status \& Development Roadmap}

The annotations are developed in collaboration with a Sanskrit Acharya to ensure philological accuracy. The complete 8-step architecture is designed; current implementation priorities are:

\textbf{Steps 1--3 (Implemented)}: Profession selection, glossary testing, and grammar certification with five annotated Yoga Sutras demonstrating the core pedagogical approach.

\textbf{Steps 4--5 (In Development)}: Automated scripture selection algorithms based on profession-difficulty matrices, mentor matching system connecting learners with university professors and private Acharyas, and native language translation modules for 10+ languages.

\textbf{Steps 6--7 (Designed)}: Integration with learning management systems (LMS) for live/recorded classes, university API connections for examination scheduling and transcript recording, and progressive credentialing workflows (Certificate → Diploma → Bachelor's degree).

\textbf{Step 8 (Planned)}: Career pathway portal with employment listings, research program applications, and advanced course recommendations.

\textbf{Content Expansion}: Extending to all 196 Yoga Sutras, then to \textit{Bhagavad G\={\i}t\={a}}, Upanishads, and other scriptures. Integrating comparative commentaries (Vyasa, Vacaspati Misra, Bhoja Raja) with toggle functionality.

\section{Conclusion}

\textit{Ashta Padi} demonstrates how profession-based personalization combined with formal university credentialing can create accessible yet rigorous pathways to Sanskrit scholarship. By structuring learning from domain-specific terminology through progressive certification (Certificate → Diploma → Bachelor's → PhD), we address both the linguistic barriers and credentialing needs that have limited scripture study to traditional \textit{gurukula} settings. The current demo validates the foundational pedagogy (Steps 1--3); the complete architecture integrates mentor assignment, live instruction, and university examination systems to provide a comprehensive alternative to conventional Sanskrit education. This work shows that computational tools can preserve traditional scholarship while democratizing access through structured, credentialed learning pathways. The demo is available at: \url{http://your-demo-url.com}.

\section*{Acknowledgements}

We thank our Sanskrit Acharya for expert philological annotations and guidance, and the ISCLS community for feedback on this work.

\bibliographystyle{acl}
\begin{thebibliography}{}

\bibitem[\protect\citename{Goyal and Huet}2012]{goyal2012sanskrit}
Pawan Goyal and G{\'e}rard Huet.
\newblock 2012.
\newblock Design and analysis of a lean interface for Sanskrit corpus annotation.
\newblock In \textit{Proceedings of COLING 2012: Posters}, pages 177--186.

\bibitem[\protect\citename{Huet}2003]{huet2003zen}
G{\'e}rard Huet.
\newblock 2003.
\newblock Towards computational processing of Sanskrit.
\newblock In \textit{International Conference on Natural Language Processing (ICON)}, pages 40--69.

\end{thebibliography}

\end{document}
